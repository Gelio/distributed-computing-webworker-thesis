\documentclass[a4paper,11pt,twoside]{report}
% KOMPILOWAĆ ZA POMOCĄ pdfLaTeXa, PRZEZ XeLaTeXa MOŻE NIE BYĆ POLSKICH ZNAKÓW

% -------------- Kodowanie znaków, język polski -------------

\usepackage[utf8]{inputenc}
\usepackage[MeX]{polski}
\usepackage[T1]{fontenc}
\usepackage[english,polish]{babel}

\usepackage{amsmath, amsfonts, amsthm, latexsym} % głównie symbole matematyczne, środowiska twierdzeń

\usepackage[final]{pdfpages} % inputowanie pdfa
\usepackage[backend=bibtex, style=verbose-trad2]{biblatex}


% ---------------- Marginesy, akapity, interlinia ------------------

\usepackage[inner=20mm, outer=20mm, bindingoffset=10mm, top=25mm, bottom=25mm]{geometry}


\linespread{1.5}
\allowdisplaybreaks

\usepackage{indentfirst} % opcjonalnie; pierwszy akapit z wcięciem
\setlength{\parindent}{5mm}


%--------------------------- ŻYWA PAGINA ------------------------

\usepackage{fancyhdr}
\pagestyle{fancy}
\fancyhf{}
% numery stron: lewa do lewego, prawa do prawego 
\fancyfoot[LE,RO]{\thepage} 
% prawa pagina: zawartość \rightmark do lewego, wewnętrznego (marginesu) 
\fancyhead[LO]{\sc \nouppercase{\rightmark}}
% lewa pagina: zawartość \leftmark do prawego, wewnętrznego (marginesu) 
\fancyhead[RE]{\sc \leftmark}

\renewcommand{\chaptermark}[1]{
\markboth{\thechapter.\ #1}{}}

% kreski oddzielające paginy (górną i dolną):
\renewcommand{\headrulewidth}{0 pt} % 0 - nie ma, 0.5 - jest linia


\fancypagestyle{plain}{% to definiuje wygląd pierwszej strony nowego rozdziału - obecnie tylko numeracja
  \fancyhf{}%
  \fancyfoot[LE,RO]{\thepage}%
  
  \renewcommand{\headrulewidth}{0pt}% Line at the header invisible
  \renewcommand{\footrulewidth}{0.0pt}
}



% ---------------- Nagłówki rozdziałów ---------------------

\usepackage{titlesec}
\titleformat{\chapter}%[display]
  {\normalfont\Large \bfseries}
  {\thechapter.}{1ex}{\Large}

\titleformat{\section}
  {\normalfont\large\bfseries}
  {\thesection.}{1ex}{}
\titlespacing{\section}{0pt}{30pt}{20pt} 
%\titlespacing{\co}{akapit}{ile przed}{ile po} 
    
\titleformat{\subsection}
  {\normalfont \bfseries}
  {\thesubsection.}{1ex}{}


% ----------------------- Spis treści ---------------------------
\def\cleardoublepage{\clearpage\if@twoside
\ifodd\c@page\else\hbox{}\thispagestyle{empty}\newpage
\if@twocolumn\hbox{}\newpage\fi\fi\fi}


% kropki dla chapterów
\usepackage{etoolbox}
\makeatletter
\patchcmd{\l@chapter}
  {\hfil}
  {\leaders\hbox{\normalfont$\m@th\mkern \@dotsep mu\hbox{.}\mkern \@dotsep mu$}\hfill}
  {}{}
\makeatother

\usepackage{titletoc}
\makeatletter
\titlecontents{chapter}% <section-type>
  [0pt]% <left>
  {}% <above-code>
  {\bfseries \thecontentslabel.\quad}% <numbered-entry-format>
  {\bfseries}% <numberless-entry-format>
  {\bfseries\leaders\hbox{\normalfont$\m@th\mkern \@dotsep mu\hbox{.}\mkern \@dotsep mu$}\hfill\contentspage}% <filler-page-format>

\titlecontents{section}
  [1em]
  {}
  {\thecontentslabel.\quad}
  {}
  {\leaders\hbox{\normalfont$\m@th\mkern \@dotsep mu\hbox{.}\mkern \@dotsep mu$}\hfill\contentspage}

\titlecontents{subsection}
  [2em]
  {}
  {\thecontentslabel.\quad}
  {}
  {\leaders\hbox{\normalfont$\m@th\mkern \@dotsep mu\hbox{.}\mkern \@dotsep mu$}\hfill\contentspage}
\makeatother



% ---------------------- Spisy tabel i obrazków ----------------------

\renewcommand*{\thetable}{\arabic{chapter}.\arabic{table}}
\renewcommand*{\thefigure}{\arabic{chapter}.\arabic{figure}}
%\let\c@table\c@figure % jeśli włączone, numeruje tabele i obrazki razem


% --------------------- Definicje, twierdzenia etc. ---------------


\makeatletter
\newtheoremstyle{definition}%    % Name
{3ex}%                          % Space above
{3ex}%                          % Space below
{\upshape}%                      % Body font
{}%                              % Indent amount
{\bfseries}%                     % Theorem head font
{.}%                             % Punctuation after theorem head
{.5em}%                            % Space after theorem head, ' ', or \newline
{\thmname{#1}\thmnumber{ #2}\thmnote{ (#3)}}%  % Theorem head spec (can be left empty, meaning `normal')
\makeatother

% ----------------------------- POLSKI --------------------------------

\theoremstyle{definition}
\newtheorem{theorem}{Twierdzenie}[chapter]
\newtheorem{lemma}[theorem]{Lemat}
\newtheorem{example}[theorem]{Przykład}
\newtheorem{proposition}[theorem]{Stwierdzenie}
\newtheorem{corollary}[theorem]{Wniosek}
\newtheorem{definition}[theorem]{Definicja}
\newtheorem{remark}[theorem]{Uwaga}



% ----------------------------- Dowód -----------------------------

%\makeatletter
%\renewenvironment{proof}[1][\proofname]
%{\par
%  \vspace{-12pt}% remove the space after the theorem
%  \pushQED{\qed}%
%  \normalfont
%  \topsep0pt \partopsep0pt % no space before
%  \trivlist
%  \item[\hskip\labelsep
%        \sc
%    #1\@addpunct{:}]\ignorespaces
%}
%{%
%  \popQED\endtrivlist\@endpefalse
%  \addvspace{20pt} % some space after
%}
%
%\renewcommand{\qedhere}{\hfill \qedsymbol}
%\makeatother





% -------------------------- POCZĄTEK --------------------------


% --------------------- Ustawienia użytkownika ------------------

\newcommand{\tytul}{Tytuł pracy dyplomowej w języku polskim}
\renewcommand{\title}{English title}
\newcommand{\type}{inżyniers} % magisters, licencjac
\newcommand{\supervisor}{dr inż. Promotor X}



\begin{document}
\sloppy

\includepdf[pages=-]{titlepage}


% ------------------ STRONA Z PODPISAMI AUTORA/AUTORÓW I PROMOTORA ------------------


\thispagestyle{empty}\newpage
\null

\vfill

\begin{center}
\begin{tabular}[t]{ccc}

............................................. & \hspace*{100pt} & .............................................\\
podpis promotora & \hspace*{100pt} & podpis autora


\end{tabular}
\end{center}



% ---------------------------- ABSTRAKTY -----------------------------
% W PRACY PO POLSKU, NAPIERW STRESZCZENIE PL, POTEM ABSTRACT EN

{
\begin{abstract}

\begin{center}
\tytul
\end{center}

Streszczam.

Lorem ipsum dolor sit amet, consetetur sadipscing elit, sed diam nonumyeirmod tempor invidunt ut labore et dolore magna aliquyam erat, sed diamvoluptua. At vero eos et accusam et justo duo dolores et ea rebum. Stet clita kasd gubergren, no sea takimata sanctus est Lorem ipsum dolor sit amet.\\

\noindent \textbf{Słowa kluczowe:} slowo1, slowo2, ...
\end{abstract}
}

\null\thispagestyle{empty}\newpage

{\selectlanguage{english}
\begin{abstract}

\begin{center}
\title
\end{center}

Lorem ipsum dolor sit amet, consetetur sadipscing elitr, sed diam nonumyeirmod tempor invidunt ut labore et dolore magna aliquyam erat, sed diamvoluptua. At vero eos et accusam et justo duo dolores et ea rebum. Stet clita kasd gubergren, no sea takimata sanctus est Lorem ipsum dolor sit amet.

Lorem ipsum dolor sit amet, consetetur sadipscing elitr, sed diam nonumyeirmod tempor invidunt ut labore et dolore magna aliquyam erat, sed diamvoluptua. At vero eos et accusam et justo duo dolores et ea rebum. Stet clita kasd gubergren, no sea takimata sanctus est Lorem ipsum dolor sit amet.\\

\noindent \textbf{Keywords:} keyword1, keyword2, ...
\end{abstract}
}


% --------------------- OŚWIADCZENIE -----------------------------------------


\null\thispagestyle{empty}\newpage

\null \hfill Warszawa, dnia ..................\\

\par\vspace{5cm}

\begin{center}
Oświadczenie
\end{center}

\indent Oświadczam, że pracę \type ką pod
tytułem ,,\tytul '', której promotorem jest \supervisor , wykonałam/wykonałem
samodzielnie, co poświadczam własnoręcznym podpisem.
\vspace{2cm}


\begin{flushright}
  \begin{minipage}{50mm}
    \begin{center}
      ..............................................

    \end{center}
  \end{minipage}
\end{flushright}

\thispagestyle{empty}
\newpage

\null\thispagestyle{empty}\newpage


% ------------------- 4. Spis treści ---------------------
\pagenumbering{gobble}
\tableofcontents
\thispagestyle{empty}

\newpage % JEŻELI SPIS TREŚCI MA PARZYSTĄ LICZBĘ STRON, ZAKOMENTOWAĆ
% ALBO JAK KTOŚ WOLI WTEDY DWIE STRONY ODSTĘPU, DODAĆ \null\newpage

% -------------- 5. ZASADNICZA CZĘŚĆ PRACY --------------------
\null\thispagestyle{empty}\newpage
\pagestyle{fancy}
\pagenumbering{arabic}
\setcounter{page}{11} % JEŻELI Z POWODU DUŻEJ ILOŚCI STRON W SPISIE TREŚCI SIĘ NIE ZGADZA, TRZEBA ZMODYFIKOWAĆ RĘCZNIE

\chapter{Wstęp}
    Opis czego będzie dotyczyła praca inżynierska

\chapter{Analiza problemu}
    \section{Cel projektu}
        Opisanie, że aktualnie nie ma na rynku rozwiązań pozwalających na obliczenia rozproszone w przeglądarce.
    
    \section{Opis istniejących rozwiązań}
        Blazor, Bridge.NET, JSIL, inne frameworki do obliczeń rozproszonych
        Zaznaczenie, że my spinamy te rzeczy w jedno
        
    \section{Wymagania funkcjonalne}
        Jak na dokumentacji z PZ
        
    \section{Wymagania niefunkcjonalne}
        Jak na dokumentacji z PZ
    
\chapter{Architektura systemu}
    Sekcje i treść jak na dokumentacji z PZ
    Rozdział opisujący system z poziomu zaplanowanej architektury

    \section{Projekt modułów}
        \subsection{Serwer}
        \subsection{Panel administracyjny}
        \subsection{Węzły obliczeniowe}
    
    \section{Szczegółowy opis implementacji}
        \subsection{Serwer}
        \subsection{Biblioteka do implementacji algorytmów równoległych}
        \subsection{Panel administracyjny}
        \subsection{Węzły obliczeniowe}
    
    \section{Zabezpieczenia}
        \subsection{Uwierzytelnianie}
        \subsection{Ograniczenia przy zwracaniu wyników}
        \subsection{Poziom zaufania węzłów}

\chapter{Opis aplikacji}
    Rozdział opisujący to, co stworzyliśmy
    
    \section{Wykorzystane biblioteki}
        \subsection{Biblioteki serwera aplikacyjnego}
        \subsection{Biblioteki serwera udostępniającego interfejs użytkownika}
    
    \section{Platforma docelowa}
        Wymagania itp.
    
    \section{Instrukcje}
        \subsection{Instrukcja instalacji}
        \subsection{Instrukcja wdrożenia}
        \subsection{Instrukcja uruchomienia}
        \subsection{Instrukcja użycia}
            W wersji skróconej, a nie takiej jak na dokumentacji z PZ
    
    \section{Testy akceptacyjne}
    
    \section{Raport z testów akceptacyjnych}

\chapter{Podsumowanie}
    \section{Własności rozwiązania}
        Benchmark wydajności C\# w WASM vs C\# natywnie
    
    \section{Możliwości na rozwój aplikacji}
    
    \section{Problemy implementacyjne}
        Głównie problemy z mono-wasm i zaskoczenie z Dockerem na Arch Linux
    
    \section{Wnioski z projektu}
        W tej sekcji zostaną przedstawione wnioski wyciągnięte z implementacji projektu.
        
        \subsection{Współpraca z eksperymentalnymi technologiami}
            Z uwagi na eksperymentalną naturę projektu \textit{mono-wasm}, jego dokumentacja była niewielka. Dostępny był jeden przykład, za pomocą którego można było rzeczywiście uruchomić metodę z biblioteki DLL zbudowanej z kodu C\#, natomiast jak okazało się później, nie było to wystarczające do pełnego funkcjonowania systemu tworzonego w ramach pracy inżynierskiej.
            
            W tej sytuacji wymagany był kontakt z osobami tworzącymi projekt \textit{mono-wasm} oraz czytanie ich otwartego kodu źródłowego.
            
            Sam interfejs udostępniany przez projekt \textit{mono-wasm} wydał nam się zawiły. Wymagał on używania zmiennych globalnych oraz wywoływania metod z mało przyjaznymi wartościami parametrów.
            
            Brak dokumentacji i większej ilości przykładów sprawił, że nad tą częścią pracy inżynierskiej spędziliśmy spędzili więcej czasu, niż planowaliśmy.
            
            W przyszłości wykorzystując eksperymentalne biblioteki warto już na samym starcie dokładnie sprawdzić dokumentację oraz przykłady użycia do tego stopnia, żeby zweryfikować czy wszystkie wymagania projektu będą możliwe do zrealizowania. W razie wątpliwości na początku skontaktować się z twórcami czy poszukać informacji na Internecie zamiast robić to podczas prób implementacji.
            
            W tym projekcie wydajność obliczeń na węzłach nie była jednym z wymagań, ale gdyby tak było, to należałoby ją zweryfikować na początku projektu, aby uniknąć niespodzianek w przyszłości. W aktualnym stanie nie moglibyśmy znacznie przyśpieszyć obliczeń w przeglądarce ze względu na wykorzystanie do tego zewnętrznej biblioteki.
        
        \subsection{Tworzenie interfejsu użytkownika jest czasochłonne}
            Planując harmonogram projektu na początku tworzenia pracy inżynierskiej nie wzięliśmy pod uwagę, że część dotycząca interfejsu użytkownika będzie tak czasochłonna, jak okazała się w rzeczywistości.
            
            Przy interfejsie użytkownika warto wziąć pod uwagę obsługę błędów. Często jest ona pomijana, a jest jednym z kluczowych praktyk poprawiających zadowolenie użytkownika z korzystania z systemu (ang. \textit{User Experience}).
            
            W przypadku podobnie wyglądających tabel, formularzy i innych komponentów warto jest poświęcić czas na stworzenie ogólnych i łatwych do skonfigurowania komponentów, które będą zawierały całą logikę w sobie, a następnie prostego użycia ich w wielu miejscach na stronie.
            
            Oprócz spójnego wyglądu tychże komponentów wszystkie zmiany logiki są wymagane wyłącznie w jednym miejscu i nie ma ryzyka, że pewne miejsce w aplikacji zostanie pominięte podczas wprowadzania modyfikacji.
            
            Jesteśmy zadowoleni z takiego podejścia zastosowanego przez nas w tej pracy inżynierskiej i stojąc przed podobnym zadaniem w przyszłości ponownie z niego skorzystamy.
        
        \subsection{Integracja w środowisku wdrożeniowym}
            Podczas uruchomienia projektu w środowisku wdrożeniowym mogą pojawić się niespodziewane problemy. Tak było w przypadku tej pracy inżynierskiej.
            
            Z uwagi na fakt, że integrację w środowisku wdrożeniowym rozpoczęliśmy na miesiąc przed oddaniem projektu, problemy te można było rozwiązać w spokoju i zgodnie z zaleceniami.
            
            Próbując uruchomić system po raz pierwszy w środowisku wdrożeniowym niedługo przed oddaniem projektu mogłoby się okazać, że powstałych problemów nie da się rozwiązać w krótkim czasie, więc oddanie projektu by się opóźniło.
            
            Warto zatem próbować uruchomić system na jak największej ilości środowisk docelowych jak najwcześniej.
            
        \subsection{Dokładna weryfikacja wymagań}
            Jednym z początkowych wymagań było uruchamianie kodu użytkownika w środowisku izolowanym na serwerze tak, aby nie mógł wyrządzić on niepożądanych działań w systemie ani nie mógł odczytać plików systemowych.
            
            Zweryfikowaliśmy na początku projektu, że takie rozwiązanie jest dostępne w \textit{.NET Standard 2.0}. Wykorzystując \textit{.NET Core 2.2}, który implementuje \textit{.NET Standard 2.0} wywnioskowaliśmy, że będziemy mogli z niego skorzystać w naszym projekcie.
            
            Okazało się jednak, że mimo, że takie rozwiązanie (\texttt{AppDomain}) jest w teorii w \textit{.NET Standard 2.0}, to nie jest wspierane w \textit{.NET Core}, zatem musieliśmy zrezygnować z tego wymagania.
            
            Na przyszłość warto pamiętać, aby w pełni zweryfikować wymagania na wstępie, szczególnie jeżeli będzie się wykorzystywało mechanizmy, z którymi nie ma się do czynienia na porządku dziennym.
        


% -------------------- 6. Bibliografia -----------------------
% Bibliografia leksykograficznie wg nazwisk autorów
% Dla ambitnych - można skorzystać z BibTeX-a


\begin{thebibliography}{25}%jak ktoś ma więcej książek, to niech wpisze większą liczbę
	% \bibitem[numerek]{referencja} Autor, \emph{Tytuł}, Wydawnictwo, rok, strony
	% cytowanie: \cite{referencja1, referencja 2,...}
	\bibitem[1]{jsonapi} Dokumentacja \emph{JSON API} \url{https://jsonapi.org/}.
	\bibitem[2]{mono} Projektu \emph{Mono} \url{https://www.mono-project.com/}.
	\bibitem[3]{mono-wasm} Repozytorium projektu \emph{mono-wasm} \url{https://github.com/mono/mono/tree/master/sdks/wasm}.
	\bibitem[4]{dotnet-core} Środowisko \emph{.NET Core} \url{https://www.microsoft.com/net/download}.
	\bibitem[5]{aspnet-core} Tworzenie aplikacji internetowej \emph{ASP.NET Core} \url{https://docs.microsoft.com/pl-pl/aspnet/core/tutorials/first-mvc-app/?view=aspnetcore-2.1}.
	\bibitem[6]{jest} Dokumentacja biblioteki do testowania \emph{jest} \url{https://jestjs.io/docs/en/getting-started}.
	\bibitem[7]{formik} Repozytorium biblioteki do tworzenia formularzy \emph{formik} \url{https://github.com/jaredpalmer/formik}.
	\bibitem[8]{react-table} Dokumentacja biblioteki do tworzenia tabel \emph{react-table} \url{https://react-table.js.org/#/story/readme}.
	\bibitem[9]{react} Poradnik do biblioteki \emph{React} \url{https://reactjs.org/tutorial/tutorial.html}.
	\bibitem[10]{typescript} Język \emph{Typescript} w 5 minut \url{https://www.typescriptlang.org/docs/handbook/typescript-in-5-minutes.html}.
	\bibitem[11]{postgresql} Poradnik obsługi bazy danych \emph{PostgreSQL} \url{http://www.postgresqltutorial.com/}.
	\bibitem[12]{ef-core} Poradnik do biblioteki \emph{Entity Framework Core} \url{http://www.entityframeworktutorial.net/efcore/entity-framework-core.aspx}.
	\bibitem[13]{jsonapi-dotnet-core} Repozytorium biblioteki \emph{JSON API .NET Core} \url{https://github.com/json-api-dotnet/JsonApiDotNetCore}.
	\bibitem[14]{webassembly} Specyfikacja \emph{WebAssembly} \url{https://webassembly.github.io/spec/}.
	\bibitem[15]{next.js} Dokumentacja technologii \emph{next.js} \url{https://nextjs.org/docs}.
	\bibitem[16]{evergreen} Dokumentacja biblioteki \emph{evergreen} \url{https://evergreen.segment.com/components/}.
	\bibitem[17]{kitsu} Repozytorium biblioteki \emph{kitsu} \url{https://github.com/wopian/kitsu/tree/master/packages/kitsu#readme}.
	\bibitem[18]{dotnet-standard} Repozytorium ze specyfikacją \emph{.NET Standard 2.0} \url{https://github.com/dotnet/standard}.
	\bibitem[19]{mvvm} Opis wzorca \emph{MVVM} w połączeniu z frameworkiem \emph{React}. \url{https://medium.cobeisfresh.com/level-up-your-react-architecture-with-mvvm-a471979e3f21}.
	\bibitem[20]{licencja-mono} Licencja \emph{Mono} \url{https://github.com/mono/mono/blob/master/LICENSE}
	\bibitem[21]{licencja-postgresql} Licencja \emph{PostgreSQL} \url{https://github.com/npgsql/Npgsql.EntityFrameworkCore.PostgreSQL/blob/dev/LICENSE}
	\bibitem[22]{licencja-nunit} Licencja \emph{NUnit} \url{https://github.com/nunit/docs/wiki/License}
\end{thebibliography}

\thispagestyle{empty}
\pagenumbering{gobble}



% --- 7. Wykaz symboli i skrótów - jeśli nie ma, zakomentować
\chapter*{Wykaz symboli i skrótów}

\begin{tabular}{p{.15\textwidth} p{.85\textwidth}}
	API 
	& Application Programming Interface, w kontekście tego projektu jest to udostępnienie możliwości komunikacji panelu administracyjnego z serwerem za pomocą protokołu HTTP i formatu JSON \\
	JSON
	& Javascript Object Notation, format serializacji danych przypominające objekty języka Javascript \\
	MIT
	& Massachusetts Institute of Technology, użyte w tym dokumencie w kontekście licencji wolnego oprogramowania dającej możliwość pełnego używania, kopiowania i modyfikacji produktu. Licencja dostępna jest pod adresem \newline
	\url{https://opensource.org/licenses/MIT} \\
	Apache 2.0
	& Licencja wolnego oprogramowania pozwalająca na używanie, modyfikację i redystrybucję produktu objętego licencją. Licencja dostępna jest pod adresem \newline
	\url{https://www.apache.org/licenses/LICENSE-2.0} \\
	BSD-3-Clause
	& Licencja wolnego oprogramowania pozwalająca na używanie, modyfikację i redystrybucję produktu objętego licencją. Licencja dostępna jest pod adresem \newline
	\url{https://opensource.org/licenses/BSD-3-Clause}
\end{tabular}
\\
\thispagestyle{empty}


% ----- 8. Spis rysunków - jeśli nie ma, zakomentować --------
\listoffigures
\thispagestyle{empty}
Jak nie występują, usunąć.


% ------------ 9. Spis tabel - jak wyżej ------------------
\renewcommand{\listtablename}{Spis tabel}
\listoftables
\thispagestyle{empty}
Jak nie występują, usunąć.


% 10. Spis załączników - jak nie ma załączników, to zakomentować lub usunąć

\chapter*{Spis załączników}

Informacje o dołączonych dokumentacjach frontendu, projektu C\# i API.

\begin{enumerate}
\item Załącznik 1
\item Załącznik 2
\item Jak nie występują, usunąć rozdział.
\end{enumerate}
\thispagestyle{empty}


\end{document}