\documentclass[a4paper,11pt,twoside]{report}
% KOMPILOWAĆ ZA POMOCĄ pdfLaTeXa, PRZEZ XeLaTeXa MOŻE NIE BYĆ POLSKICH ZNAKÓW

\usepackage{longtable}
\usepackage{makecell}

% -------------- Kodowanie znaków, język polski -------------

\usepackage[utf8]{inputenc}
\usepackage[MeX]{polski}
\usepackage[T1]{fontenc}
\usepackage[english,polish]{babel}

\usepackage{float}

\usepackage{url}


\usepackage{amsmath, amsfonts, amsthm, latexsym} % głównie symbole matematyczne, środowiska twierdzeń

\usepackage[final]{pdfpages} % inputowanie pdfa
%\usepackage[backend=bibtex, style=verbose-trad2]{biblatex}


% ---------------- Marginesy, akapity, interlinia ------------------


\usepackage[inner=20mm, outer=20mm, bindingoffset=10mm, top=25mm, bottom=25mm]{geometry}

\renewcommand{\tablename}{Tabela}

\linespread{1.5}
\allowdisplaybreaks

\usepackage{indentfirst} % opcjonalnie; pierwszy akapit z wcięciem
\setlength{\parindent}{5mm}


%--------------------------- ŻYWA PAGINA ------------------------

\setlength{\headheight}{14pt}
\usepackage{fancyhdr}
\pagestyle{fancy}
\fancyhf{}
% numery stron: lewa do lewego, prawa do prawego 
\fancyfoot[LE,RO]{\thepage} 
% prawa pagina: zawartość \rightmark do lewego, wewnętrznego (marginesu) 
\fancyhead[LO]{\sc \nouppercase{\rightmark}}
% lewa pagina: zawartość \leftmark do prawego, wewnętrznego (marginesu) 
\fancyhead[RE]{\sc \leftmark}

\renewcommand{\chaptermark}[1]{
\markboth{\thechapter.\ #1}{}}

% kreski oddzielające paginy (górną i dolną):
\renewcommand{\headrulewidth}{0 pt} % 0 - nie ma, 0.5 - jest linia


\fancypagestyle{plain}{% to definiuje wygląd pierwszej strony nowego rozdziału - obecnie tylko numeracja
  \fancyhf{}%
  \fancyfoot[LE,RO]{\thepage}%
  
  \renewcommand{\headrulewidth}{0pt}% Line at the header invisible
  \renewcommand{\footrulewidth}{0.0pt}
}

% ---------------- Kod ------------------------------------
\usepackage{alltt}

% ---------------- Nagłówki rozdziałów ---------------------

\usepackage{titlesec}
\titleformat{\chapter}%[display]
  {\normalfont\Large \bfseries}
  {\thechapter.}{1ex}{\Large}

\titleformat{\section}
  {\normalfont\large\bfseries}
  {\thesection.}{1ex}{}
\titlespacing{\section}{0pt}{30pt}{20pt} 
%\titlespacing{\co}{akapit}{ile przed}{ile po} 
    
\titleformat{\subsection}
  {\normalfont \bfseries}
  {\thesubsection.}{1ex}{}


% ----------------------- Spis treści ---------------------------
\def\cleardoublepage{\clearpage\if@twoside
\ifodd\c@page\else\hbox{}\thispagestyle{empty}\newpage
\if@twocolumn\hbox{}\newpage\fi\fi\fi}


% kropki dla chapterów
\usepackage{etoolbox}
\makeatletter
\patchcmd{\l@chapter}
  {\hfil}
  {\leaders\hbox{\normalfont$\m@th\mkern \@dotsep mu\hbox{.}\mkern \@dotsep mu$}\hfill}
  {}{}
\makeatother

\usepackage{titletoc}
\makeatletter
\titlecontents{chapter}% <section-type>
  [0pt]% <left>
  {}% <above-code>
  {\bfseries \thecontentslabel.\quad}% <numbered-entry-format>
  {\bfseries}% <numberless-entry-format>
  {\bfseries\leaders\hbox{\normalfont$\m@th\mkern \@dotsep mu\hbox{.}\mkern \@dotsep mu$}\hfill\contentspage}% <filler-page-format>

\titlecontents{section}
  [1em]
  {}
  {\thecontentslabel.\quad}
  {}
  {\leaders\hbox{\normalfont$\m@th\mkern \@dotsep mu\hbox{.}\mkern \@dotsep mu$}\hfill\contentspage}

\titlecontents{subsection}
  [2em]
  {}
  {\thecontentslabel.\quad}
  {}
  {\leaders\hbox{\normalfont$\m@th\mkern \@dotsep mu\hbox{.}\mkern \@dotsep mu$}\hfill\contentspage}
\makeatother


% Dodawanie pdfów
\usepackage{pdfpages}


% ---------------------- Spisy tabel i obrazków ----------------------

\renewcommand*{\thetable}{\arabic{chapter}.\arabic{table}}
\renewcommand*{\thefigure}{\arabic{chapter}.\arabic{figure}}
%\let\c@table\c@figure % jeśli włączone, numeruje tabele i obrazki razem


% --------------------- Definicje, twierdzenia etc. ---------------


\makeatletter
\newtheoremstyle{definition}%    % Name
{3ex}%                          % Space above
{3ex}%                          % Space below
{\upshape}%                      % Body font
{}%                              % Indent amount
{\bfseries}%                     % Theorem head font
{.}%                             % Punctuation after theorem head
{.5em}%                            % Space after theorem head, ' ', or \newline
{\thmname{#1}\thmnumber{ #2}\thmnote{ (#3)}}%  % Theorem head spec (can be left empty, meaning `normal')
\makeatother

% ----------------------------- POLSKI --------------------------------

\theoremstyle{definition}
\newtheorem{theorem}{Twierdzenie}[chapter]
\newtheorem{lemma}[theorem]{Lemat}
\newtheorem{example}[theorem]{Przykład}
\newtheorem{proposition}[theorem]{Stwierdzenie}
\newtheorem{corollary}[theorem]{Wniosek}
\newtheorem{definition}[theorem]{Definicja}
\newtheorem{remark}[theorem]{Uwaga}



% ----------------------------- Dowód -----------------------------

%\makeatletter
%\renewenvironment{proof}[1][\proofname]
%{\par
%  \vspace{-12pt}% remove the space after the theorem
%  \pushQED{\qed}%
%  \normalfont
%  \topsep0pt \partopsep0pt % no space before
%  \trivlist
%  \item[\hskip\labelsep
%        \sc
%    #1\@addpunct{:}]\ignorespaces
%}
%{%
%  \popQED\endtrivlist\@endpefalse
%  \addvspace{20pt} % some space after
%}
%
%\renewcommand{\qedhere}{\hfill \qedsymbol}
%\makeatother





% -------------------------- POCZĄTEK --------------------------


% --------------------- Ustawienia użytkownika ------------------

\newcommand{\tytul}{System obliczeń rozproszonych z węzłami działającymi w przeglądarce}
\renewcommand{\title}{English title}
\newcommand{\type}{inżyniers} % magisters, licencjac
\newcommand{\supervisor}{mgr inż. Jan Karwowski}



\begin{document}
\sloppy

\includepdf[pages=-]{titlepage}


% ------------------ STRONA Z PODPISAMI AUTORA/AUTORÓW I PROMOTORA ------------------


% ---------------------------- ABSTRAKTY -----------------------------
% W PRACY PO POLSKU, NAPIERW STRESZCZENIE PL, POTEM ABSTRACT EN

{
\begin{abstract}

\begin{center}
\tytul
\end{center}

Dokument zawiera dokumentację powykonawczą systemu przygotowanego w ramach tworzonej przez nas pracy inżynierskiej. Zawiera ona opis architektury, wymagania systemowe, użyte biblioteki, instrukcję instalacji, uruchomienia, użycia oraz utrzymania, a także raport odstępstw od specyfikacji i scenariusz testów akceptacyjnych.
\end{abstract}
}


% --------------------- OŚWIADCZENIE -----------------------------------------


% ------------------- 4. Spis treści ---------------------
\pagenumbering{gobble}
\tableofcontents
\thispagestyle{empty}

% \newpage % JEŻELI SPIS TREŚCI MA PARZYSTĄ LICZBĘ STRON, ZAKOMENTOWAĆ
% ALBO JAK KTOŚ WOLI WTEDY DWIE STRONY ODSTĘPU, DODAĆ \null\newpage

% -------------- 5. ZASADNICZA CZĘŚĆ PRACY --------------------
\null\thispagestyle{empty}\newpage
\pagestyle{fancy}
\pagenumbering{arabic}
\setcounter{page}{4} % JEŻELI Z POWODU DUŻEJ ILOŚCI STRON W SPISIE TREŚCI SIĘ NIE ZGADZA, TRZEBA ZMODYFIKOWAĆ RĘCZNIE
        
\chapter{Wstęp}
    Opis czego będzie dotyczyła praca inżynierska

\chapter{Analiza problemu}
    \section{Cel projektu}
        Opisanie, że aktualnie nie ma na rynku rozwiązań pozwalających na obliczenia rozproszone w przeglądarce.
    
    \section{Opis istniejących rozwiązań}
        Blazor, Bridge.NET, JSIL, inne frameworki do obliczeń rozproszonych
        Zaznaczenie, że my spinamy te rzeczy w jedno
        
    \section{Wymagania funkcjonalne}
        Jak na dokumentacji z PZ
        
    \section{Wymagania niefunkcjonalne}
        Jak na dokumentacji z PZ
    
\chapter{Architektura systemu}
    Sekcje i treść jak na dokumentacji z PZ
    Rozdział opisujący system z poziomu zaplanowanej architektury

    \section{Projekt modułów}
        \subsection{Serwer}
        \subsection{Panel administracyjny}
        \subsection{Węzły obliczeniowe}
    
    \section{Szczegółowy opis implementacji}
        \subsection{Serwer}
        \subsection{Biblioteka do implementacji algorytmów równoległych}
        \subsection{Panel administracyjny}
        \subsection{Węzły obliczeniowe}
    
    \section{Zabezpieczenia}
        \subsection{Uwierzytelnianie}
        \subsection{Ograniczenia przy zwracaniu wyników}
        \subsection{Poziom zaufania węzłów}

\chapter{Opis aplikacji}
    Rozdział opisujący to, co stworzyliśmy
    
    \section{Wykorzystane biblioteki}
        \subsection{Biblioteki serwera aplikacyjnego}
        \subsection{Biblioteki serwera udostępniającego interfejs użytkownika}
    
    \section{Platforma docelowa}
        Wymagania itp.
    
    \section{Instrukcje}
        \subsection{Instrukcja instalacji}
        \subsection{Instrukcja wdrożenia}
        \subsection{Instrukcja uruchomienia}
        \subsection{Instrukcja użycia}
            W wersji skróconej, a nie takiej jak na dokumentacji z PZ
    
    \section{Testy akceptacyjne}
    
    \section{Raport z testów akceptacyjnych}

\chapter{Podsumowanie}
    \section{Własności rozwiązania}
        Benchmark wydajności C\# w WASM vs C# natywnie
    
    \section{Możliwości na rozwój aplikacji}
    
    \section{Problemy implementacyjne}
        Głównie problemy z mono-wasm i zaskoczenie z Dockerem na Arch Linux
    
    \section{Wnioski z projektu}
        
\chapter{Bibliografia}

\chapter{Wykaz skrótów}

\chapter{Załączniki}
    Informacje o dołączonych dokumentacjach frontendu, projektu C\# i API.

\end{document}
