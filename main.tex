\documentclass[a4paper,11pt,twoside]{report}
% KOMPILOWAĆ ZA POMOCĄ pdfLaTeXa, PRZEZ XeLaTeXa MOŻE NIE BYĆ POLSKICH ZNAKÓW

% -------------- Kodowanie znaków, język polski -------------

\usepackage[utf8]{inputenc}
\usepackage[MeX]{polski}
\usepackage[T1]{fontenc}
\usepackage[english,polish]{babel}

\usepackage{amsmath, amsfonts, amsthm, latexsym} % głównie symbole matematyczne, środowiska twierdzeń

\usepackage[final]{pdfpages} % inputowanie pdfa
\usepackage[backend=bibtex, style=verbose-trad2]{biblatex}


% ---------------- Marginesy, akapity, interlinia ------------------

\usepackage[inner=20mm, outer=20mm, bindingoffset=10mm, top=25mm, bottom=25mm]{geometry}


\linespread{1.5}
\allowdisplaybreaks

\usepackage{indentfirst} % opcjonalnie; pierwszy akapit z wcięciem
\setlength{\parindent}{5mm}

%--------------------------- Tabele -----------------------------
\usepackage{longtable}
\usepackage{makecell}
\renewcommand{\tablename}{Tabela}


%-------------------------- Kod źródłowy ------------------------
\usepackage{alltt}

%--------------------------- ŻYWA PAGINA ------------------------

\usepackage{fancyhdr}
\pagestyle{fancy}
\fancyhf{}
% numery stron: lewa do lewego, prawa do prawego 
\fancyfoot[LE,RO]{\thepage} 
% prawa pagina: zawartość \rightmark do lewego, wewnętrznego (marginesu) 
\fancyhead[LO]{\sc \nouppercase{\rightmark}}
% lewa pagina: zawartość \leftmark do prawego, wewnętrznego (marginesu) 
\fancyhead[RE]{\sc \leftmark}

\renewcommand{\chaptermark}[1]{
\markboth{\thechapter.\ #1}{}}

% kreski oddzielające paginy (górną i dolną):
\renewcommand{\headrulewidth}{0 pt} % 0 - nie ma, 0.5 - jest linia


\fancypagestyle{plain}{% to definiuje wygląd pierwszej strony nowego rozdziału - obecnie tylko numeracja
  \fancyhf{}%
  \fancyfoot[LE,RO]{\thepage}%
  
  \renewcommand{\headrulewidth}{0pt}% Line at the header invisible
  \renewcommand{\footrulewidth}{0.0pt}
}



% ---------------- Nagłówki rozdziałów ---------------------

\usepackage{titlesec}
\titleformat{\chapter}%[display]
  {\normalfont\Large \bfseries}
  {\thechapter.}{1ex}{\Large}

\titleformat{\section}
  {\normalfont\large\bfseries}
  {\thesection.}{1ex}{}
\titlespacing{\section}{0pt}{30pt}{20pt} 
%\titlespacing{\co}{akapit}{ile przed}{ile po} 
    
\titleformat{\subsection}
  {\normalfont \bfseries}
  {\thesubsection.}{1ex}{}


% ----------------------- Spis treści ---------------------------
\def\cleardoublepage{\clearpage\if@twoside
\ifodd\c@page\else\hbox{}\thispagestyle{empty}\newpage
\if@twocolumn\hbox{}\newpage\fi\fi\fi}


% kropki dla chapterów
\usepackage{etoolbox}
\makeatletter
\patchcmd{\l@chapter}
  {\hfil}
  {\leaders\hbox{\normalfont$\m@th\mkern \@dotsep mu\hbox{.}\mkern \@dotsep mu$}\hfill}
  {}{}
\makeatother

\usepackage{titletoc}
\makeatletter
\titlecontents{chapter}% <section-type>
  [0pt]% <left>
  {}% <above-code>
  {\bfseries \thecontentslabel.\quad}% <numbered-entry-format>
  {\bfseries}% <numberless-entry-format>
  {\bfseries\leaders\hbox{\normalfont$\m@th\mkern \@dotsep mu\hbox{.}\mkern \@dotsep mu$}\hfill\contentspage}% <filler-page-format>

\titlecontents{section}
  [1em]
  {}
  {\thecontentslabel.\quad}
  {}
  {\leaders\hbox{\normalfont$\m@th\mkern \@dotsep mu\hbox{.}\mkern \@dotsep mu$}\hfill\contentspage}

\titlecontents{subsection}
  [2em]
  {}
  {\thecontentslabel.\quad}
  {}
  {\leaders\hbox{\normalfont$\m@th\mkern \@dotsep mu\hbox{.}\mkern \@dotsep mu$}\hfill\contentspage}
\makeatother



% ---------------------- Spisy tabel i obrazków ----------------------

\renewcommand*{\thetable}{\arabic{chapter}.\arabic{table}}
\renewcommand*{\thefigure}{\arabic{chapter}.\arabic{figure}}
%\let\c@table\c@figure % jeśli włączone, numeruje tabele i obrazki razem


% --------------------- Definicje, twierdzenia etc. ---------------


\makeatletter
\newtheoremstyle{definition}%    % Name
{3ex}%                          % Space above
{3ex}%                          % Space below
{\upshape}%                      % Body font
{}%                              % Indent amount
{\bfseries}%                     % Theorem head font
{.}%                             % Punctuation after theorem head
{.5em}%                            % Space after theorem head, ' ', or \newline
{\thmname{#1}\thmnumber{ #2}\thmnote{ (#3)}}%  % Theorem head spec (can be left empty, meaning `normal')
\makeatother

% ----------------------------- POLSKI --------------------------------

\theoremstyle{definition}
\newtheorem{theorem}{Twierdzenie}[chapter]
\newtheorem{lemma}[theorem]{Lemat}
\newtheorem{example}[theorem]{Przykład}
\newtheorem{proposition}[theorem]{Stwierdzenie}
\newtheorem{corollary}[theorem]{Wniosek}
\newtheorem{definition}[theorem]{Definicja}
\newtheorem{remark}[theorem]{Uwaga}



% ----------------------------- Dowód -----------------------------

%\makeatletter
%\renewenvironment{proof}[1][\proofname]
%{\par
%  \vspace{-12pt}% remove the space after the theorem
%  \pushQED{\qed}%
%  \normalfont
%  \topsep0pt \partopsep0pt % no space before
%  \trivlist
%  \item[\hskip\labelsep
%        \sc
%    #1\@addpunct{:}]\ignorespaces
%}
%{%
%  \popQED\endtrivlist\@endpefalse
%  \addvspace{20pt} % some space after
%}
%
%\renewcommand{\qedhere}{\hfill \qedsymbol}
%\makeatother





% -------------------------- POCZĄTEK --------------------------


% --------------------- Ustawienia użytkownika ------------------

\newcommand{\tytul}{Tytuł pracy dyplomowej w języku polskim}
\renewcommand{\title}{English title}
\newcommand{\type}{inżyniers} % magisters, licencjac
\newcommand{\supervisor}{dr inż. Promotor X}



\begin{document}
\sloppy

\includepdf[pages=-]{titlepage}


% ------------------ STRONA Z PODPISAMI AUTORA/AUTORÓW I PROMOTORA ------------------


\thispagestyle{empty}\newpage
\null

\vfill

\begin{center}
\begin{tabular}[t]{ccc}

............................................. & \hspace*{100pt} & .............................................\\
podpis promotora & \hspace*{100pt} & podpis autora


\end{tabular}
\end{center}



% ---------------------------- ABSTRAKTY -----------------------------
% W PRACY PO POLSKU, NAPIERW STRESZCZENIE PL, POTEM ABSTRACT EN

{
\begin{abstract}

\begin{center}
\tytul
\end{center}

Streszczam.

Lorem ipsum dolor sit amet, consetetur sadipscing elit, sed diam nonumyeirmod tempor invidunt ut labore et dolore magna aliquyam erat, sed diamvoluptua. At vero eos et accusam et justo duo dolores et ea rebum. Stet clita kasd gubergren, no sea takimata sanctus est Lorem ipsum dolor sit amet.\\

\noindent \textbf{Słowa kluczowe:} slowo1, slowo2, ...
\end{abstract}
}

\null\thispagestyle{empty}\newpage

{\selectlanguage{english}
\begin{abstract}

\begin{center}
\title
\end{center}

Lorem ipsum dolor sit amet, consetetur sadipscing elitr, sed diam nonumyeirmod tempor invidunt ut labore et dolore magna aliquyam erat, sed diamvoluptua. At vero eos et accusam et justo duo dolores et ea rebum. Stet clita kasd gubergren, no sea takimata sanctus est Lorem ipsum dolor sit amet.

Lorem ipsum dolor sit amet, consetetur sadipscing elitr, sed diam nonumyeirmod tempor invidunt ut labore et dolore magna aliquyam erat, sed diamvoluptua. At vero eos et accusam et justo duo dolores et ea rebum. Stet clita kasd gubergren, no sea takimata sanctus est Lorem ipsum dolor sit amet.\\

\noindent \textbf{Keywords:} keyword1, keyword2, ...
\end{abstract}
}


% --------------------- OŚWIADCZENIE -----------------------------------------


\null\thispagestyle{empty}\newpage

\null \hfill Warszawa, dnia ..................\\

\par\vspace{5cm}

\begin{center}
Oświadczenie
\end{center}

\indent Oświadczam, że pracę \type ką pod
tytułem ,,\tytul '', której promotorem jest \supervisor , wykonałam/wykonałem
samodzielnie, co poświadczam własnoręcznym podpisem.
\vspace{2cm}


\begin{flushright}
  \begin{minipage}{50mm}
    \begin{center}
      ..............................................

    \end{center}
  \end{minipage}
\end{flushright}

\thispagestyle{empty}
\newpage

\null\thispagestyle{empty}\newpage


% ------------------- 4. Spis treści ---------------------
\pagenumbering{gobble}
\tableofcontents
\thispagestyle{empty}

\newpage % JEŻELI SPIS TREŚCI MA PARZYSTĄ LICZBĘ STRON, ZAKOMENTOWAĆ
% ALBO JAK KTOŚ WOLI WTEDY DWIE STRONY ODSTĘPU, DODAĆ \null\newpage

% -------------- 5. ZASADNICZA CZĘŚĆ PRACY --------------------
\null\thispagestyle{empty}\newpage
\pagestyle{fancy}
\pagenumbering{arabic}
\setcounter{page}{11} % JEŻELI Z POWODU DUŻEJ ILOŚCI STRON W SPISIE TREŚCI SIĘ NIE ZGADZA, TRZEBA ZMODYFIKOWAĆ RĘCZNIE

\chapter{Wstęp}
    Opis czego będzie dotyczyła praca inżynierska

\chapter{Analiza problemu}
    \section{Cel projektu}
        Opisanie, że aktualnie nie ma na rynku rozwiązań pozwalających na obliczenia rozproszone w przeglądarce.
    
    \section{Opis istniejących rozwiązań}
        Blazor, Bridge.NET, JSIL, inne frameworki do obliczeń rozproszonych
        Zaznaczenie, że my spinamy te rzeczy w jedno
        
    \section{Wymagania funkcjonalne}
        Jak na dokumentacji z PZ
        
    \section{Wymagania niefunkcjonalne}
        Jak na dokumentacji z PZ
    
\chapter{Architektura systemu}
    Sekcje i treść jak na dokumentacji z PZ
    Rozdział opisujący system z poziomu zaplanowanej architektury

    \section{Projekt modułów}
        \subsection{Serwer}
        \subsection{Panel administracyjny}
        \subsection{Węzły obliczeniowe}
    
    \section{Szczegółowy opis implementacji}
        \subsection{Serwer}
        \subsection{Biblioteka do implementacji algorytmów równoległych}
        \subsection{Panel administracyjny}
        \subsection{Węzły obliczeniowe}
    
    \section{Zabezpieczenia}
        \subsection{Uwierzytelnianie}
        \subsection{Ograniczenia przy zwracaniu wyników}
        \subsection{Poziom zaufania węzłów}

\chapter{Opis aplikacji}
    Poniższy rozdział opisuje przygotowany produkt. Na potrzeby tego rozdziału zostały utworzone: lista wykorzystywanych bibliotek wraz z ich licencjami, instrukcje przydatne podczas pracy z opisywanym systemem oraz lista testów akceptacyjnych wraz z ich rezultatami.
    
    \section{Wykorzystane biblioteki}
         W tej sekcji zostaną przedstawione wykorzystane biblioteki z podziałem na część serwera aplikacyjnego i część serwera panela administracyjnego.
        
        \subsection{Biblioteki serwera aplikacyjnego}
        Tabela \ref{biblioteki-serwer} zawiera biblioteki użyte w serwerze aplikacyjnym wraz z ich licencjami.
        
        \begin{center}
            \begin{longtable}{| p{.10\textwidth} | p{.20\textwidth} | p{.40\textwidth} | p{.15\textwidth} |}
                \caption{Wykorzystane biblioteki w serwerze aplikacyjnym wraz z ich licencjami}
                \label{biblioteki-serwer} \\
                \hline
                nr. & Nazwa i wersja & Opis & Licencja \\ \hline
                \endfirsthead
                \multicolumn{4}{c}{\tablename\ \thetable\ -- \textit{Kontynuowane z poprzedniej strony}} \\
                \hline
                nr. & Nazwa i wersja & Opis & Licencja \\ \hline
                \endhead
                
                1 & ASP.NET Core v2.2.0 & Framework do tworzenia serwerów API. & Apache 2.0 \\ \hline
                2 & JSON API .NET Core v2.4.0 & Biblioteka pozwalająca na tworzenie kontrolerów API zgodnych ze standardem JSON API. & MIT \\ \hline
                3 & Entity Framework Core v2.2.0 & Biblioteka pozwlająca na łatwą komunikację z bazą danych z poziomu kodu. & Apache 2.0 \\ \hline
                4 & Npgsql EntityFrameworkCore PostrgreSQL v2.2.0 & Biblioteka pozwalająca na komunikację z bazą danych PostgreSQL. & PostgreSQL \cite{licencja-postgresql} \\ \hline
                5 & Mono v5.16.0 & Implementacja .NET Framework działająca w systemie UNIX. & Własna licencja Mono \cite{licencja-mono} \\ \hline
                6 & mono-wasm v5.16.0 & Interpreter języka C\# działający w technologii WebAssembly. & Własna licencja Mono \cite{licencja-mono} \\ \hline
                7 & NUnit 3.11.0 & Biblioteka do tworzenia i uruchamiania testów jednostkowych. & NUnit License \cite{licencja-nunit} \\ \hline
                8 & Moq v4.10.1 & Biblioteka pozwalająca na tworzenie zaślepek podczas testów jednostkowych. & BSD-3-Clause \\ \hline
            \end{longtable}
        \end{center}
        
        \subsection{Biblioteki serwera udostępniającego interfejs użytkownika}
        Tabela \ref{biblioteki-frontend} zawiera biblioteki użyte w serwerze budującym i wyświetlającym interfejs użytkownika wraz z ich licencjami.
        
        \begin{center}
            \begin{longtable}{| p{.10\textwidth} | p{.20\textwidth} | p{.40\textwidth} | p{.15\textwidth} |}
                \caption{Wykorzystane biblioteki w serwerze interfejsu użytkownika wraz z ich licencjami}
                \label{biblioteki-frontend} \\
                \hline
                nr. & Nazwa i wersja & Opis & Licencja \\ \hline
                \endfirsthead
                \multicolumn{4}{c}{\tablename\ \thetable\ -- \textit{Kontynuowane z poprzedniej strony}} \\
                \hline
                nr. & Nazwa i wersja & Opis & Licencja \\ \hline
                \endhead
                
                1 & date-fns v1.30.1 & Biblioteka do łatwego zarządzania datami & MIT \\ \hline
                2 & evergreen-ui v4.0.2 & Biblioteka z komponentami interfejsu użytkownika & MIT \\ \hline
                3 & express v4.16.4 & Biblioteka pozwalająca na stworzenie serwera HTTP w Node.js & MIT \\ \hline
                4 & formik v1.3.1 & Biblioteka do zarządzania formularzami w aplikacji & MIT \\ \hline
                5 & immutable v4.0.0-rc.11 & Biblioteka dostarczająca niezmienne struktury danych & MIT \\ \hline
                6 & kitsu v6.4.0 & Biblioteka do komunikacji z JSON API & MIT \\ \hline
                7 & next v7.0.2 & Framework do budowania aplikacji w Javascripcie umożliwiający Server-Side Rendering & MIT \\ \hline
                8 & normalize.css v8.0.1 & Biblioteka od normalizacji stylowania komponentów w przeglądarce & MIT \\ \hline
                9 & path-to-regexp v2.4.0 & Biblioteka pozwalająca na sprawdzenie czy URL pasuje do danej ścieżki na stronie & MIT \\ \hline
                10 & ramda v0.25.0 & Biblioteka udostępniająca wiele pomocniczych funkcji & MIT \\ \hline
                11 & React v16.5.2 & Framework do wyświetlania interfejsów użytkownika & MIT \\ \hline
                12 & React-table v6.8.6 & Biblioteka udostępniająca komponent tabeli & MIT \\ \hline
                13 & yup v0.26.6 & Biblioteka udostępniająca narzędzia do budowania walidatorów danych & MIT \\ \hline
                14 & jest v23.6.0 & Biblioteka umożliwiająca tworzenie i uruchamianie testów jednostkowych & MIT \\ \hline
            \end{longtable}
        \end{center}
    
    \section{Platforma docelowa}
        Do instalacji i uruchomienia systemu potrzebny wymagany jest system o następujących parametrach:
        
        \begin{itemize}
            \item System operacyjny Arch Linux 
            \item Minimalnie 2 GB pamięci RAM
            \item Minimalnie 10 GB pamięci dyskowej
        \end{itemize}
        
        Do uruchomienia systemu potrzebne będą następujące programy:
        
        \begin{itemize}
            \item Docker (w wersji minimalnie 18.09.0)
            \item Docker-compose (w wersji minimalnie 1.23.2)
        \end{itemize}
    
    \section{Instrukcje}
        Poniższy rozdział zawiera instrukcje dotyczące instalacji, wdrożenia, uruchomienia, użycia oraz utrzymania systemu.
    
    \subsection{Instrukcja instalacji}
        W celu instalacji wymaganych zależności należy wywołać następujące komendy:

        \begin{alltt}
    # pacman -S docker
    # systemctl start docker
    # pacman -S docker-compose
        \end{alltt}

        W celu weryfikacji poprawnej instalacji należy wywołać komendy:
        \begin{alltt}
    # docker version
    # docker-compose -v
        \end{alltt}

        Jeśli powyższe komendy poprawnie wyświetlą zainstalowaną wersję, to znaczy, że instalacja się powiodła. 

    \subsection{Instrukcja wdrożenia}
    
        Jeśli projekt jest uruchamiany na serwerze, który nie posiada ograniczeń w dostępie do sieci, to można przejść do podpunktu \ref{start-system}. W przeciwnym wypadku należy zbudować obrazy \textit{Docker} na maszynie z dostępem do sieci, a następnie skopiować przygotowane obrazy na maszynę docelową, na której system ma być uruchomiony.
        
        Zakładamy, że katalog zawierający kod źródłowy projektu nazwany jest \texttt{distributed-computing-webworker}. Jeżeli będzie nazwany w inny sposób, poniższe komendy powinny zostać zmodyfikowane, zamieniając \texttt{distributed-computing-webworker} w nazwach obrazów na nazwę katalogu, w którym znajduje się kod źródłowy projektu.
        
        W celu zbudowania obrazów \textit{Docker} należy wywołać następujące komendy na maszynie budującej:

        \begin{alltt}
    # docker-compose -f docker-compose.yml -f docker-compose.prod.yml \textbackslash
    up --build --no-start
    # docker save distributed-computing-webworker_backend \textbackslash
    distributed-computing-webworker_frontend postgres \textbackslash 
    nginx | gzip > docker_images.tar.gz    
        \end{alltt}

        Następnym krokiem jest skopiowanie wyeksportowanych obrazów oraz plików \textit{docker-compose.yml}, \textit{docker-compose.prod.yml} na maszynę decelową. Można to wykonać na przykład wykorzystując narzędzie \textit{scp}:

        \begin{alltt}
    # scp docker_images.tar.gz docker-compose.yml docker-compose.prod.yml \textbackslash
    username@production:~
        \end{alltt}

        gdzie \texttt{username} jest nazwą konta użytkownika w systemie maszyny docelowej, a \texttt{production} jest adresem maszyny docelowej.
        
        Następnie należy zaimportować skopiowane obrazy do \textit{Docker} uruchomionego na maszynie docelowej wywołując na niej następującą komendę:

        \begin{alltt}
    # gunzip -c ~/docker_images.tar.gz | docker load
        \end{alltt}



    \subsection{Instrukcja uruchomienia}
        \label{start-system}
        Aby uruchomić projekt należy z katalogu zawierającego pliki \textit{docker-compose.yml}, \textit{docker-compose.prod.yml} wywołać komendę:

        \begin{alltt}
    # docker-compose -f docker-compose.yml -f docker-compose.prod.yml up -d
        \end{alltt}

        Jeśli w systemie nie znajdują się zbudowane obrazy \textit{Docker}, to podczas trwania tej komendy zostaną one zbudowane.
        
        Po zakończeniu wykonywania komendy działanie systemu można zweryfikować przez próbę wejścia na adres \url{http://[adres_serwera]}. Jeśli uruchomienie systemu się powiodło, to powinna zostać wyświetlona strona zawierająca informacje o projekcie.

    \subsection{Instrukcja utrzymania}

        Poniższa sekcja zawiera informacje dotyczące zarządzania stanem całego systemu oraz opis tworzenia kopii zapasowej i przywracania jej.

        \subsubsection{Usunięcie systemu}

            Aby skasować system należy użyć komendy:

            \begin{alltt}
    docker-compose -f docker-compose.yml -f docker-compose.prod.yml down
            \end{alltt}

            Spowoduje to usunięcie wszystkich kontenerów, na których działał system, wraz z wszelkimi danymi zapisanymi w systemie.

        \subsubsection{Zatrzymanie systemu}

            Aby zatrzymać system zachowując dane należy użyć komendy:
            
            \begin{alltt}
    docker-compose -f docker-compose.yml -f docker-compose.prod.yml stop
            \end{alltt}

            Ponowne uruchomienie systemu należy wykonać zgodnie z punktem \ref{start-system}.

        \subsubsection{Tworzenie kopii zapasowej}

            Aby utworzyć kopię zapasową należy uruchomić skrypt \textit{backup.sh} znajdujący się w katalogu \textit{scripts} umieszczonym w głównym folderze projektu.
            
            Kopia bazy danych oraz definicji zadań zostanie zapisana w pliku \textit{backup\_data.tar.gz}

        \subsubsection{Przywracanie kopii zapasowej}

            Aby przywrócić kopię zapasową należy uruchomić skrypt \textit{restore.sh} podając jako argument nazwę archiwium zawierającego kopię zapasową. Skrypt \textit{restore.sh} znajduje się w katalogu \textit{scripts} w głównym folderze projektu.

    \subsection{Instrukcja użycia}
        W poniższej sekcji przedstawiono instrukcję użycia produktu z podziałem na etap logowania, dodawania definicji zadania, dodawania zadania, obliczania zadania oraz pobierania wyników.
        
        \subsubsection{Logowanie}
        
        Po uruchomieniu systemu interfejs użytkownika dostępny jest pod adresem \url{http://[adres_serwera]}. Dostęp do wykonywania obliczeń dostępny jest dla wszystkich użytkowników, a panel administracyjny wymaga zalogowania.
        Domyślne dane do logowania do panelu administracyjnego:

        \begin{alltt}
    login: admin
    hasło: D1stributed\$
        \end{alltt}


        Po pierwszym zalogowaniu należy zmienić domyślne hasło. Opcja zmiany hasła jest dostępna z poziomu paska bocznego.
        
        \subsubsection{Dodanie definicji zadania}
        \label{distributed-task-definition-add-guide}
        
        Aby dodać nową definicję zadania należy przejść do zakładki \texttt{Distributed Task Definitions}.
        
        Po przejściu do zakładki z definicjami zadań zostanie wyświetlona tabela zawierająca zadania dodane już do systemu. W prawym górnym rogu znajduje się ikona plusa na zielonym tle, po kliknięciu w nią nastąpi przekierowanie do formularza dodawania nowej definicji zadania.
        
        
        Aby dodać nową definicję zadania należy wypełnić wyświetlony formularz.
        
        
        W polu \textit{MainDLL} należy umieścić bibliotekę implementującą interfejs \textit{IProblemPlugin} z biblioteki \textit{DistributedComputingLibrary}. Pole \textit{AdditionalDlls} powinno dostarczyć wszystkie pozostałe biblioteki znajdujące się folderze budowy definicji zadania. Po wypełnieniu wszystkich pól należy wysłać formularz klikająć przycisk \texttt{Submit}.
        
        Przykładowe zadania dostępne są w folderze \textit{example-problems} znajdującym się w głównym katalogu projektu.
        
        
        \subsubsection{Dodanie nowego zadania}
        \label{distributed-task-add-guide}
        
        Po dodaniu nowej definicji zadania następuje przekierowanie do strony wyświetlającej szczegóły dodanego zadania. 
        Z tej strony możemy zlecić obliczenia nowych danych wejściowych. W tym celu należy kliknąć na ikonę plusa na zielonym tle znajdującego się nad tabelą z zadaniami.
        
        
        W celu zlecenia wykonania zadania należy wypełnić widoczny formularz. 
        
        
        Pole \textit{Priority} określa priorytet zadania, im jest wyższy, tym szybciej zostanie ono przekazane do wykonania.
        
        Pole \textit{Trust level to complete} określa wymaganą sumę poziomów zaufania węzłów obliczeniowych wykonujących każde z podzadań, aby móc je zakończyć. 
        
        W polu \textit{Task input} należy umieścić plik zawierający dane wejściowe.
        
        Po wypełnieniu wszystkich pól należy wysłać formularz klikająć w przycisk \texttt{Submit}. Po kliknięciu nastąpi przekierowanie do detali utworzonego zadania. 
          
        
        W skład detali wchodzą informacje o utworzonym zadaniu oraz status poszczególnych podzadań. W górnej części widocznej strony znajdują się również przciski pozwalające na zarządzanie zadaniem oraz pobranie danych wejściowych.
        
        \subsubsection{Rozpoczęcie obliczeń}
        
        Aby obliczenia się rozpoczęły do systemu musi dołączyć conajmniej jeden węzęł obliczeniowy. Aby urządzenie chcące zostać węzłem obliczeniowym dołączyło do systemu, musi ono wejść do zakładki \texttt{Worker} dostępnej w pasku bocznym, a następnie kliknąć na przycisk \texttt{Start the worker}. Po kliknięciu w przycisk urządzenie zarejestruje się jako węzeł obliczeniowy i zaczynie odpytywać o nowe zadania. Węzły sieciowe mogą w każdym momencie przerwać obliczenia przez kliknięcie w przycisk \texttt{Stop the worker} oraz dowolnie konfigurować ilość udostępnianych wątków. 
        
        
        Stan węzła obliczeniowego jest dostępny w sekcji \textit{Worker information}, a status użyczonych wątków jest udostępniony do wglądu w tabeli \textit{Individual thread statuses}. W momencie, gdy stan węzła zawiera informację o braku dostępnych zadań do obliczenia wiemy, że nasze zadanie zostało wykonane.
        
        
        \subsubsection{Sprawdzenie wyników}
        
        Aby sprawdzić rezultat zleconego zadania należy wybrać opcję \texttt{Distributed Tasks} w panelu bocznym. Po przejściu do sekcji ze zleconymi zadaniami zostanie wyświetlona strona zawierająca zadania znajdujące się już w systemie. Aby sprawdzić wynik dodanego wcześniej zadania należy kliknąć w przycisk \texttt{See details} w wierszu zawierającym to zadanie.
        
        
        Po przekierowaniu do strony zawierającej detale zadania można pobrać plik wynikowy klikając na przycisk \texttt{Download results}.

    
    \section{Testy akceptacyjne}
        
        Testy akceptacyjne zostały podzielone na dwie kategorie ze względu na poziom uprawnień użytkownika.
        
        \subsection{Użytkownik anonimowy}
            Poniższe testy dotyczą użytkownika, który nie jest zalogowany w systemie.
        
        
            \subsubsection{Zalogowanie} 
                Użytkownik uzyskuje pełen dostęp do systemu (zostaje mu przyznany poziom uprawnień \textit{Administrator}). 
            
                \texttt{Lista kroków}:
                \begin{enumerate}
                    \item Przejście do strony logowania
                    \item Podanie poprawnych danych logowania w formularzu
                    \item Wysłanie formularza
                \end{enumerate}
            
            \subsubsection{Uruchomienie węzła obliczeniowego} 
                Użytkownik zostaje zarejestrowany jako węzęł obliczeniowy i zaczyna odpytywać serwer 
                
                \texttt{Lista kroków}:
                \begin{enumerate}
                    \item Przejście do panelu węzła obliczeniowego
                    \item Kliknięcie w przycisk \texttt{Start the worker}
                \end{enumerate}
            
            \subsubsection{Zatrzymanie węzła obliczeniowego}  
                Wykonywanie obliczeń zostaje anulowane, a użytkownik może swobodnie opuścić stronę węzła obliczeniowego.
    
                \texttt{Lista kroków}:
                \begin{enumerate}
                    \item Przejście do panelu węzła obliczeniowego
                    \item Kliknięcie w przycisk \texttt{Start the worker}
                    \item Kliknięcie w przycisk \texttt{Stop the worker}
                \end{enumerate}

            \subsubsection{Zwiększenie ilości wątków}   
                Węzeł obliczeniowy do obliczania zadań będzie używał o jednego wątku więcej niż poprzednio. 

                \texttt{Lista kroków}:
                \begin{enumerate}
                    \item Przejście do panelu węzła obliczeniowego
                    \item Kliknięcie w przycisk \texttt{Start the worker}
                    \item Kliknięcie w przycisk z ikoną plusa
                \end{enumerate}

            \subsubsection{Zmniejszenie ilości wątków}   
                Jeśli usuwany wątek wykonywał obliczenie, to system zgłasza do serwera jego anulowanie. Węzeł obliczeniowy do obliczania zadań bedzie teraz używał o jednego wątku mniej niż poprzednio (chyba, że poprzednio używał tylko 1 wątku).

                \texttt{Lista kroków}:
                \begin{enumerate}
                    \item Przejście do panelu węzła obliczeniowego
                    \item Kliknięcie w przycisk \texttt{Start the worker}
                    \item Kliknięcie w przycisk z ikoną minusa
                \end{enumerate}

        \subsection{Administrator}
            Poniższe testy są wykonywane z poziomem uprawnień administratora.

            \subsubsection{Dodanie definicji zadania}   
                Zadanie zostaje zapisane w systemie i jest widoczne w tabeli z zadaniami. 

                \texttt{Lista kroków}:
                \begin{enumerate}
                    \item Przejście do strony z tabelą zawierającą definicje zadania
                    \item Kliknięcie w ikonę plusa
                    \item Wypełnienie formularza zgodnie z \ref{distributed-task-definition-add-guide}
                    \item Wysłanie formularza
                \end{enumerate}

            \subsubsection{Informacje o zadaniu}    
                Wyświetlona strona zawiera informacje o zadaniu wraz z postępem wykonywania przez węzły obliczeniowe. 
                
                \texttt{Lista kroków}:
                \begin{enumerate}
                    \item Przejście do tabeli z zadaniami
                    \item Kliknięcie w przycisk \texttt{See details} przy dowolnym zadaniu
                \end{enumerate}

            \subsubsection{Lista definicji zadań}  
                Wyświetlona strona zawiera listę definicji zadań wraz z ich nazwą oraz odnośnikiem do szczegółowych informacji o każdej z definicji.

                \texttt{Lista kroków}:
                \begin{enumerate}
                    \item Przejście do tabeli z definicjami zadań
                \end{enumerate}

            \subsubsection{Lista dodanych zadań}   
                Wyświetlona strona zawiera listę zadań wraz z ich nazwą, odnośnikiem do definicji zadania oraz odnośnikiem do szczegółów zadania.
                
                \texttt{Lista kroków}:
                \begin{enumerate}
                    \item Przejście do tabeli z zadaniami
                \end{enumerate}

            \subsubsection{Usunięcie zadania}  
                Zadanie wraz z podzadaniami zostają usunięte z systemu.
                
                \texttt{Lista kroków}:
                \begin{enumerate}
                    \item Przejście do tabeli z zadaniami
                    \item Kliknięcie w przycisk \texttt{Delete} przy zadaniu
                \end{enumerate}

            \subsubsection{Usunięcie definicji zadania}    
                Definicja zadania wraz z zadaniami i podzadaniami zostają usunięte z systemu.
                
                \texttt{Lista kroków}:
                \begin{enumerate}
                    \item Przejście do tabeli z definicjami zadań
                    \item Kliknięcie w przycisk \texttt{Delete} przy definicji zadania
                \end{enumerate}

            \subsubsection{Pobieranie wyników} 
                Po kliknięciu w przycisk zostaje pobrany plik zawierający
                wynik zadania.
                
                \texttt{Lista kroków}:
                \begin{enumerate}
                    \item Przejście do tabeli z zadaniami
                    \item Kliknięcie w przycisk \texttt{See details} przy zadaniu o statusie \textit{Done}
                    \item Kliknięcie w przycisk \texttt{Download results}
                \end{enumerate}

            \subsubsection{Pobieranie danych wejściowych}  
                Po kliknięciu w przycisk zostaje pobrany plik zawierający dane wejściowe zadania.
                
                \texttt{Lista kroków}:
                \begin{enumerate}
                    \item Przejście do tabeli z zadaniami
                    \item Kliknięcie w przycisk \texttt{See details} przy zadaniu
                    \item Kliknięcie w przycisk \texttt{Download input data}
                \end{enumerate}

            \subsubsection{Edycja zadania}  
                Informacje o zadaniu zostają zmienione.
                
                \texttt{Lista kroków}:
                \begin{enumerate}
                    \item Przejście do tabeli z zadaniami
                    \item Kliknięcie w przycisk \texttt{Edit} przy zadaniu
                    \item Wypełnienie oraz wysłanie formularza
                \end{enumerate}

            \subsubsection{Edycja definicji zadania}    
                Informacje o definicji zadania zostają zmienione.
                
                \texttt{Lista kroków}:
                \begin{enumerate}
                    \item Przejście do tabeli z definicjami zadań
                    \item Kliknięcie w przycisk \texttt{Edit} przy definicji zadania
                    \item Wypełnienie oraz wysłanie formularza
                \end{enumerate}

            \subsubsection{Wylogowanie}
                Użytkownik zostaje wylogowany i przestaje mieć status administratora.

                \texttt{Lista kroków}:
                \begin{enumerate}
                    \item Kliknięcie w przycisk \texttt{Logout} z panelu bocznego
                \end{enumerate}

            \subsubsection{}    
            
    \section{Raport z testów akceptacyjnych}
    
        Wszystkie testy akceptacyjne zostały zakończone sukcesem.

\chapter{Podsumowanie}
    \section{Własności rozwiązania}
        Benchmark wydajności C\# w WASM vs C\# natywnie
    
    \section{Możliwości na rozwój aplikacji}
    
    \section{Problemy implementacyjne}
        Głównie problemy z mono-wasm i zaskoczenie z Dockerem na Arch Linux
    
    \section{Wnioski z projektu}
        


% -------------------- 6. Bibliografia -----------------------
% Bibliografia leksykograficznie wg nazwisk autorów
% Dla ambitnych - można skorzystać z BibTeX-a


\begin{thebibliography}{25}%jak ktoś ma więcej książek, to niech wpisze większą liczbę
    % \bibitem[numerek]{referencja} Autor, \emph{Tytuł}, Wydawnictwo, rok, strony
    % cytowanie: \cite{referencja1, referencja 2,...}
    \bibitem[1]{jsonapi} Dokumentacja \emph{JSON API} \url{https://jsonapi.org/}.
    \bibitem[2]{mono} Projektu \emph{Mono} \url{https://www.mono-project.com/}.
    \bibitem[3]{mono-wasm} Repozytorium projektu \emph{mono-wasm} \url{https://github.com/mono/mono/tree/master/sdks/wasm}.
    \bibitem[4]{dotnet-core} Środowisko \emph{.NET Core} \url{https://www.microsoft.com/net/download}.
    \bibitem[5]{aspnet-core} Tworzenie aplikacji internetowej \emph{ASP.NET Core} \url{https://docs.microsoft.com/pl-pl/aspnet/core/tutorials/first-mvc-app/?view=aspnetcore-2.1}.
    \bibitem[6]{jest} Dokumentacja biblioteki do testowania \emph{jest} \url{https://jestjs.io/docs/en/getting-started}.
    \bibitem[7]{formik} Repozytorium biblioteki do tworzenia formularzy \emph{formik} \url{https://github.com/jaredpalmer/formik}.
    \bibitem[8]{react-table} Dokumentacja biblioteki do tworzenia tabel \emph{react-table} \url{https://react-table.js.org/#/story/readme}.
    \bibitem[9]{react} Poradnik do biblioteki \emph{React} \url{https://reactjs.org/tutorial/tutorial.html}.
    \bibitem[10]{typescript} Język \emph{Typescript} w 5 minut \url{https://www.typescriptlang.org/docs/handbook/typescript-in-5-minutes.html}.
    \bibitem[11]{postgresql} Poradnik obsługi bazy danych \emph{PostgreSQL} \url{http://www.postgresqltutorial.com/}.
    \bibitem[12]{ef-core} Poradnik do biblioteki \emph{Entity Framework Core} \url{http://www.entityframeworktutorial.net/efcore/entity-framework-core.aspx}.
    \bibitem[13]{jsonapi-dotnet-core} Repozytorium biblioteki \emph{JSON API .NET Core} \url{https://github.com/json-api-dotnet/JsonApiDotNetCore}.
    \bibitem[14]{webassembly} Specyfikacja \emph{WebAssembly} \url{https://webassembly.github.io/spec/}.
    \bibitem[15]{next.js} Dokumentacja technologii \emph{next.js} \url{https://nextjs.org/docs}.
    \bibitem[16]{evergreen} Dokumentacja biblioteki \emph{evergreen} \url{https://evergreen.segment.com/components/}.
    \bibitem[17]{kitsu} Repozytorium biblioteki \emph{kitsu} \url{https://github.com/wopian/kitsu/tree/master/packages/kitsu#readme}.
    \bibitem[18]{dotnet-standard} Repozytorium ze specyfikacją \emph{.NET Standard 2.0} \url{https://github.com/dotnet/standard}.
    \bibitem[19]{mvvm} Opis wzorca \emph{MVVM} w połączeniu z frameworkiem \emph{React}. \url{https://medium.cobeisfresh.com/level-up-your-react-architecture-with-mvvm-a471979e3f21}.
    \bibitem[20]{licencja-mono} Licencja \emph{Mono} \url{https://github.com/mono/mono/blob/master/LICENSE}
    \bibitem[21]{licencja-postgresql} Licencja \emph{PostgreSQL} \url{https://github.com/npgsql/Npgsql.EntityFrameworkCore.PostgreSQL/blob/dev/LICENSE}
    \bibitem[22]{licencja-nunit} Licencja \emph{NUnit} \url{https://github.com/nunit/docs/wiki/License}
\end{thebibliography}

\thispagestyle{empty}
\pagenumbering{gobble}



% --- 7. Wykaz symboli i skrótów - jeśli nie ma, zakomentować
\chapter*{Wykaz symboli i skrótów}

\begin{tabular}{p{.15\textwidth} p{.85\textwidth}}
    API 
    & Application Programming Interface, w kontekście tego projektu jest to udostępnienie możliwości komunikacji panelu administracyjnego z serwerem za pomocą protokołu HTTP i formatu JSON \\
    JSON
    & Javascript Object Notation, format serializacji danych przypominające objekty języka Javascript \\
    MIT
    & Massachusetts Institute of Technology, użyte w tym dokumencie w kontekście licencji wolnego oprogramowania dającej możliwość pełnego używania, kopiowania i modyfikacji produktu. Licencja dostępna jest pod adresem \newline
    \url{https://opensource.org/licenses/MIT} \\
    Apache 2.0
    & Licencja wolnego oprogramowania pozwalająca na używanie, modyfikację i redystrybucję produktu objętego licencją. Licencja dostępna jest pod adresem \newline
    \url{https://www.apache.org/licenses/LICENSE-2.0} \\
    BSD-3-Clause
    & Licencja wolnego oprogramowania pozwalająca na używanie, modyfikację i redystrybucję produktu objętego licencją. Licencja dostępna jest pod adresem \newline
    \url{https://opensource.org/licenses/BSD-3-Clause}
\end{tabular}
\\
\thispagestyle{empty}


% ----- 8. Spis rysunków - jeśli nie ma, zakomentować --------
\listoffigures
\thispagestyle{empty}
Jak nie występują, usunąć.


% ------------ 9. Spis tabel - jak wyżej ------------------
\renewcommand{\listtablename}{Spis tabel}
\listoftables
\thispagestyle{empty}
Jak nie występują, usunąć.


% 10. Spis załączników - jak nie ma załączników, to zakomentować lub usunąć

\chapter*{Spis załączników}

Informacje o dołączonych dokumentacjach frontendu, projektu C\# i API.

\begin{enumerate}
\item Załącznik 1
\item Załącznik 2
\item Jak nie występują, usunąć rozdział.
\end{enumerate}
\thispagestyle{empty}


\end{document}